%%%%%%%%%%%%%%%%%%%%%%%%%%%%%%%%%%%%%%%%%
% University Assignment Title Page 
% LaTeX Template
% Version 1.0 (27/12/12)
%
% This template has been downloaded from:
% http://www.LaTeXTemplates.com
%
% Original author:
% WikiBooks (http://en.wikibooks.org/wiki/LaTeX/Title_Creation)
%
% License:
% CC BY-NC-SA 3.0 (http://creativecommons.org/licenses/by-nc-sa/3.0/)
% 
% Instructions for using this template:
% This title page is capable of being compiled as is. This is not useful for 
% including it in another document. To do this, you have two options: 
%
% 1) Copy/paste everything between \begin{document} and \end{document} 
% starting at \begin{titlepage} and paste this into another LaTeX file where you 
% want your title page.
% OR
% 2) Remove everything outside the \begin{titlepage} and \end{titlepage} and 
% move this file to the same directory as the LaTeX file you wish to add it to. 
% Then add \input{./title_page_1.tex} to your LaTeX file where you want your
% title page.
%
%%%%%%%%%%%%%%%%%%%%%%%%%%%%%%%%%%%%%%%%%
%\title{Title page with logo}
%----------------------------------------------------------------------------------------
%	PACKAGES AND OTHER DOCUMENT CONFIGURATIONS
%----------------------------------------------------------------------------------------

\documentclass[11pt, spanish]{article}
\usepackage[spanish]{babel}
\selectlanguage{spanish}
\usepackage[utf8x]{inputenc}
\usepackage{amsmath}
\usepackage{graphicx}
\usepackage{subcaption}
\usepackage{verbatim}
\usepackage{fancyvrb}

\usepackage{utopia}
\pagestyle{headings}

\usepackage[colorinlistoftodos]{todonotes}
\usepackage[acronym,automake]{glossaries}
\usepackage{hyperref}
\usepackage[font=small,labelfont=bf]{caption}

\renewcommand{\baselinestretch}{1.5}


\raggedbottom                            %Evita que LaTeX distribuya los espacios en blanco sobre la página, en lugar de eso los envía al fondo

\makeglossaries
\newacronym{GTC}{GTC}{Gran Telescopio de Canarias}
\newacronym{IA-UNAM}{IA-UNAM}{Instituto de Astronom\'ia – Universidad Nacional Aut\'onoma de M\'exico}
\newacronym{PFC}{PFC}{Proyecto Fin de Carrera}
\newacronym{SSL}{SSL}{Secure Socket Layer}
\newacronym{ANSI}{ANSI}{American National Standards Institute}
\newacronym{CNRI}{CNRI}{Corporation for National Research Initiatives}
\newacronym{PSF}{PSF}{Python Software Foundation}
\newacronym{PSFL}{PSFL}{Python Software Foundation Licence}
\newacronym{FSF}{FSF}{Free Software Foundation}
\newacronym{AP}{AP}{Access Point}
\newacronym{NIST}{NIST}{Instituto Nacional de Estándares y Tecnología}
\newacronym{LR-WPAN}{LR-WPAN}{Low-Rate Wireless Personal Area Network}
\newacronym{PAM}{PAM}{Pluggable Authentication Modules}
\newacronym{IDE}{IDE}{Integrated Development Environment}
\newacronym{CA}{CA}{Certificate Authority}
\newacronym{CSR}{CSR}{Certificate Signing Request}


\begin{document}
\begin{titlepage}

\newcommand{\HRule}{\rule{\linewidth}{0.5mm}} % Defines a new command for the horizontal lines, change thickness here

\center % Center everything on the page
 
%----------------------------------------------------------------------------------------
%	HEADING SECTIONS
%----------------------------------------------------------------------------------------

\textsc{\LARGE Shidix Technologies}\\[1.2cm] % Name of your university/college
\textsc{\Large Calculadora de tiempos de exposici\'on}\\[0.4cm] % Major heading such as course name
\textsc{\large Especificaciones}\\[0.4cm] % Minor heading such as course title

%----------------------------------------------------------------------------------------
%	TITLE SECTION
%----------------------------------------------------------------------------------------

\HRule \\[0.3cm]
{ \huge \bfseries FRIDA}\\[0.4cm] % Title of your document
\HRule \\[1.2cm]
 
%----------------------------------------------------------------------------------------
%	AUTHOR SECTION
%----------------------------------------------------------------------------------------

\begin{minipage}{0.4\textwidth}
\begin{flushleft} \large
\emph{Preparado por:}\\
Daniel Jacobo D\'iaz Gonz\'alez
\end{flushle2t}
\end{minipage}
~
\begin{minipage}{0.4\textwidth}
\begin{flushright} \large
\emph{Aprobado por:} \\
Jos\'e Acosta
\end{flushright}
\end{minipage}\\[2cm]

% If you don't want a supervisor, uncomment the two lines below and remove the subsection above
%\Large \emph{Author:}\\
%John \textsc{Smith}\\[3cm] % Your name

%----------------------------------------------------------------------------------------
%	DATE SECTION
%----------------------------------------------------------------------------------------

{\large \today}\\[2cm] % Date, change the \today to a set date if you want to be precise

%----------------------------------------------------------------------------------------
%	LOGO SECTION
%----------------------------------------------------------------------------------------

\begin{figure*}[h]
    \centering
    \begin{subfigure}[b]{0.5\textwidth}
        \centering
        \includegraphics[height=1.2in]{gtc}
    \end{subfigure}%
    ~ 
    \begin{subfigure}[b]{0.5\textwidth}
        \centering
        \includegraphics[height=1.2in]{unam}
    \end{subfigure}
\end{figure*}
 
%----------------------------------------------------------------------------------------

\vfill % Fill the rest of the page with whitespace

\end{titlepage}


%\begin{abstract}
%Este documento describe el ETC de FRIDA en modo imagen y en modo IFS. Este calculador sigue una metodolog\'ia similar a la de otros instrumentos que tambi\'en utilizan \'optica adaptativa. La calculadora ha sido desarrolla utilizando python, HTML5 y CSS3, estando disponible v\'ia web.
%\end{abstract}
\printacronyms
\section{Requerimientos}
El ETC est\'a disponible v\'ia web a trav\'es de la url \url{http://frida.shidix.es}, alojada en un servidor propio del Instituto de Astrof\'isica de Canarias. El código fuente está disponible a través de un repositorio git, sujeto a control de versiones.  

La interfaz web ha sido implementada utilizando el framework Django, desarrollado en Python, junto con el framework Bootstrap, desarrollado en HTML5 y CSS3; estas herramientas son open source, y están disponibles para su descarga en \url{https://www.djangoproject.com/download/} y \url{https://getbootstrap.com/docs/3.3/getting-started#download}.

El ETC tiene dos modos independientes: modo imagen y modo IFS.

\subsection{Requerimientos primarios}
\begin{itemize}
    \item El ETC ofrece dos opciones b\'asicas para la morfolog\'ia de la fuente: fuente puntual y fuente extendida. 
    \item El brillo de la fuente se especifica como la magnitud en una banda concreta. 
    \item Se contemplan los dos modos de operaci\'on del instrumento, el modo imagen y el modo IFS.
    \item Existen ficheros de configuraci\'on a partir de los que se define la transmisi\'on y emisi\'on de la atm\'osfera terrestre.
    \item Hay tres escalas espaciales para cada uno de los modos (imagen e IFS).
    \item El ETC muestra una tabla con los valores de salida, as\'i como varias gr\'aficas con la representaci\'on de los mismos.

\end{itemize}

\subsection{Requerimientos secundarios}
\begin{itemize}
    \item Se puede utlizar como entrada un conjunto de modelos para la PSF.
    \item Se han incluido diferentes distribuciones espectrales: cuerpo negro, ley de potencia, plantillas de estrellas y plantillas de galaxias.
\end{itemize}

\section{Estructura de la Web}
La interfaz web est\'a dividida en tres pesta\~nas desde la que el usuario puede introducir los datos.
\subsection{Definici\'on de la fuente}
En esta pesta\~na se inicar\'an los par\'ametros relacionados con la fuente astron\'omica.
\begin{itemize}
    \item \textbf{Distribuci\'on espectral}. Se podr\'a escoger entre las siguientes opciones:
        \begin{itemize}
            \item \textit{Cuerpo negro}. Habr\'a que definir la temperatura.
            \item \textit{Ley de potencias}. La ley de potencia se expresa como $S_\lambda = \lambda^x$, y el usuario deber\'a indicar la $x$.
            \item \textit{Espectro estelar}. Se da una serie de platillas entre las que se puede escoger.
            \item \textit{Emisi\'on de l\'inea}. Se deben definir los par\'ametros de longitud de onda, flujo de l\'inea, unidade del flujo, velocidad, continuo y unidades del continuo.
            \item \textit{Espectro definido por el usuario}. El usuario deber\'a subir un fichero en el que vendr\'a definido el espectro.
            \item \textit{Espectros de objetos no estelares}. Se puede escoger entre varias plantillas.
        \end{itemize}
   \item \textbf{Brillo}. Habr\'a que definir los siguientes par\'ametros:
        \begin{itemize}
            \item \textit{Brillo Total}. Hay que indicar el brillo en magnitud y la banda de referencia.
            \item \textit{Extinci\'on}. 
            \item \textit{Redshift / Velocidad radial}. 
        \end{itemize}
   \item \textbf{Perfil espacial}. Habr\'a que escoger entre fuente puntual y fuente extendida uniforme.
\end{itemize}

\subsection{Condiciones de la observaci\'on y configuraci\'on del GTCAO}
En este caso son tres los par\'ametros a definir.
    \begin{itemize}
        \item \textbf{Masa de aire}.
        \item \textbf{Seeing}.
        \item \textbf{Separaci\'on de estrella gu\'ia}. Estos valores se podr\'an indicar en segundos de arco.
        \item \textbf{Brillo de la estrella gu\'ia}. La banda de referencia es la R.
    \end{itemize}

\section{Screenshots}

\begin{figure}[h]
    \centering
    \includegraphics[scale=0.25]{figs/spatial-profile}
    \label{fig:spatial}
\end{figure}

\begin{figure}[h]
    \centering
    \includegraphics[scale=0.25]{figs/brightness}
    \label{fig:brightness}
\end{figure}
\begin{figure}[h]
    \centering
    \includegraphics[scale=0.2]{figs/ifs-mode}
    \caption{Modo IFS}
    \label{fig:ifs-mode}
\end{figure}
\begin{figure}[h]
    \centering
    \includegraphics[scale=0.2]{figs/image-mode}
    \caption{Modo Imagen}
    \label{fig:image-mode}
\end{figure}

\newpage
\section{Estructura de archivos}
    \subsection{Modelo-Vista-Controlador}
    \begin{itemize}
        \item \textbf{frida/views.py}. Fichero con la definición de las acciones que gobierna la web.
        \item \textbf{frida/calculator.py}. Contine las funciones para el c\'alculo de los tiempos de exposici\'on, as\'i como la definici\'on del instrumento y los modos de observaci\'on.
        \item \textbf{frida/compute\_flux.py}.
    \end{itemize}

    \subsection{Filtros}
    En el directorio frida/includes podemos encontramos el archivo \'indice de los filtros que van a estar disponibles, denominado \textbf{\textit{filters.dat}}. Es un fichero ascii, m\'as concretametne un CSV (commar-separate values) en el que vamos a definir los filtros, uno por l\'inea, especificando los siguientes campos:
    \begin{itemize}
        \item \textbf{Name}. Etiqueta con la que se va a mostrar el filtro en el selector de la web. Es de tipo cadena, y debe ser indicado entre comillas dobles.
        \item \textbf{Code}. C\'odigo de referencia. Es de tipo cadena, y debe ser indicado entre comillas dobles. 
        \item \textbf{Transmission}. El nombre del fichero con los datos de transmisi\'on para este filtro (veremos despu\'es su formato). Se encontrar\'a en \textit{frida/filters}. Es de tipo cadena, y debe ser indicado entre comillas dobles.
        \item \textbf{cut-on}. L\'imite inferior del filtro (en micras). Es de tipo float.
        \item \textbf{cut-off}. L\'imite superior del filtro (en micras). Es de tipo float.
        \item \textbf{lambda-center}. Longitud de onda de referencia para este filtro (en micras). Es de tipo float.
        \item \textbf{dit}. Tiempo de integraci\'on que el detector usar\'a por defecto. Es de tipo float.
        \item \textbf{diff\_limit}. L\'imite de difracci\'on en mili arcsec. Es de tipo float.
    \end{itemize}

\begin{Verbatim}[fontsize=\scriptsize]
Name,Code,Transmission,cut-on,cut-off,lambda_center,dit,diff_lim
"Y","Y","filterY_cold.dat",0.960,1.086,1.020,20,24.7
"J","J","filterJ_cold.dat",1.170,1.330,1.25,20,30.3
"H","H","filterH_cold.dat",1.490,1.780,1.635,5,39.6
"Ks","Ks","filterKs_cold.dat",1.990,2.310,2.145,5,51.9
"[FeII] [1.26 micron]","NB1.26","filterNB1p26_cold.dat",1.250,1.264,1.257,60,30.4
\end{Verbatim}
    Este fichero contiene una l\'inea de cabera en la que se listan los nombres de campo; adem\'as, admite l\'ineas de comentarios, para lo cual la l\'inea debe comenzar por \#.

    Los ficheros de transmisi\'on los podemos encontrar en el directorio \textit{frida/filters}. Tambi\'en son ficheros ascii (csv); son ficheros de dos columnas, en los que la primera columna indica la longitud de onda en micras y la segunda columna la transmisi\'on para esa longitud de onda. No contienen fila de cabecera.
\begin{Verbatim}[fontsize=\scriptsize]
1.60130    0.0102336
1.60226    0.0103185
1.60322    0.0104181
1.60419    0.0103159
1.60515    0.0102234
1.60611    0.0104208
1.60707    0.0104571
1.60804    0.0106585
\end{Verbatim}

    \subsection{Rendijas}
    En el directorio frida/includes podemos encontramos el archivo \'indice de las rendijas que van a estar disponibles, denominado \textbf{\textit{gratings.dat}}. Es un fichero ascii, m\'as concretametne un CSV (commar-separate values) en el que vamos a definir las rendijas, una por l\'inea, especificando los siguientes campos:

    \begin{itemize}
        \item \textbf{Name}. Etiqueta con la que se va a mostrar el filtro en el selector de la web. Es de tipo cadena.
        \item \textbf{Description}. C\'odigo de referencia. Es de tipo cadena.
        \item \textbf{Master}. C\'odigo de fabricante.
        \item \textbf{Efficiency}. El nombre del fichero con los datos de eficiencia de la rendija (veremos despu\'es su formato). Se encontrar\'a en \textit{frida/gratings}. Es de tipo cadena.
        \item \textbf{Central\_wave}. Longitud de onda central en esta rendija (en Angstroms). Es de tipo float.
        \item \textbf{Rcenter}. Resolución en la longitud de onda central. Es de tipo float.
        \item \textbf{Dispersion}. Dispersi\'on de la rendija (en Angstroms/pixel). Es de tipo float.
        \item \textbf{Wave\_ini}. L\'imite inferior de la rendija (en Angstroms). Es de tipo float.
        \item \textbf{Wave\_end}. L\'imite superior de la rendina (en Angstroms). Es de tipo float.
        \item \textbf{diff\_limit}. L\'imite de difracci\'on en mili arcsec. Es de tipo float.
    \end{itemize}

    Este fichero contiene una l\'inea de cabera en la que se listan los nombres de campo; adem\'as, admite l\'ineas de comentarios, para lo cual la l\'inea debe comenzar por \#.

\begin{Verbatim}[fontsize=\scriptsize]
Name,Description,Master,Efficiency,Central_Wave,Rcenter,Dispersion,Wave_ini,Wave_end,diff_limit
zJ Low,z+J R~1400 - 1.15 micron,*-186R,Effic_ZJLow.csv,11250,1371,4.10,7048,15450,27.2
HK Low,H+K R~1250 - 1.97 micron,*-186R,Effic_HKLow.csv,19750,1261,7.83,11700,27800,47.8
z Med,z R~3600 - 1.025 micron,*-596R,Effic_zMed.csv,10250,3597,1.42,8700,11708,24.8
J Med,J R~6700 - 1.25 micron,*-640R,Effic_JMed.csv,12500,6701,0.93,11500,13500,30.3
H Med,H R~4300 - 1.64 micron,*-760R,Effic_HMed.csv,16350,4326,1.89,14400,18300,39.6
\end{Verbatim}

    Los ficheros de eficiencia los podemos encontrar en el directorio \textit{frida/gratings}. Tambi\'en son ficheros ascii (csv); son ficheros de dos columnas, en los que la primera columna indica la longitud de onda en micras y la segunda columna, la eficiencia para esa longitud de onda. En este caso, la primera l\'inea es una cabecera en la que podemos definir la unidad a utilizar. Veamos un ejemplo.
\begin{Verbatim}[fontsize=\scriptsize]
# schema: astropy-2.0
Wave[nm],Effic[%]
1004.11,49.3485
1027.62,53.628
1045.59,55.9055
1102.34,60.2415
1104.24,60.3395
1130.33,60.5
1158.87,61.524
1177.07,61.584
1216.43,61.3695
1223.92,61.8275
1270.72,65.852
\end{Verbatim}




\end{document}
